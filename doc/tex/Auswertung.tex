\chapter{Auswertung}

    Im folgenden Kapitel werden die Ergebnisse ausgewertet und in Relation zu den Benchmarks von VPR gesetzt.
    Zunächst werden in Kapitel (\ref{sec:result-init}) die zwei verschiedenen Initialplatzierungen miteinander
    verglichen. Des Weiteren werden in Kapitel (\ref{sec:result-costs}) die Kosten der Platzierung
    der jeweiligen Benchmarks mit den Platzierungskosten von VPR verglichen.
    Abschließend werden in Kapitel (\ref{sec:result-routing}) die Ergebnisse der vorgenommenen
    Verdrahtung durch VPR verglichen.
    Alle Benchmarks wurden dabei mit einer \textit{maxIterations} von 1000
    und einer \textit{maxRippleIterations} von 20 ausgeführt

    \section{Initialplatzierung}\label{sec:result-init}

        Die Ergebnisse der Platzierung sind unter anderem abhängig von der Initialplatzierung.
        Daher wird die in dieser Arbeit entwickelte Initialplatzierung in diesem Kapitel mit einer
        zufälligen Initialplatzierung verglichen.
        Als Vergleichsgrößen werden die Laufzeit des Gesamtalgorithmus sowie die
        Platzierungskosten verwendet. Dabei gilt für beide Größen,
        dass ein niedrigerer Wert ein besseres Endergebnis darstellt. Tabelle (\ref{tab:init})
        stellt die oben genannten Zusammenhänge tabellarisch dar. Die Differenzspalte ist so zu verstehen,
        dass ein negativer Wert ein besseres Ergebnis darstellt.
        Für die Platzierungen ist die Laufzeit jedoch nicht ausschlaggebend.
        Eine höhere Laufzeit kann in manchen Fällen in Kauf genommen werden, um damit die
        Platzierungskosten niedriger zu halten.
        \\
        Auffällig ist, dass nicht jeder Benchmark von der
        ZFT-Initialplatzierung profitiert. Wie in Kapitel (\ref{sec:init}) erwähnt, ist dies damit zu erklären,
        dass die berechnete ZFT-Position für manche Blöcke nicht exakt berechnet werden kann und somit
        die Initialplatzierung nicht ideal ist. Des Weiteren ist denkbar,
        dass die Zufallsplatzierung durch Zufall eine idealere Initialplatzierung für das iterative
        Verfahren der Kräfteplatzierung findet.
        \\
        In diesem Fall kann durch die ZFT-Initialplatzierung in 11 von 20 Fällen eine bessere
        Platzierung erzielt werden. Zu erwähnen ist auch, dass sich in 8 von 11 Fällen zusätzlich die
        Laufzeit erhöht.

        \begin{center}
            \begin{longtable}{| l | r | r | r | r | r | r |}
                \hline
                \multirow{2}*{\textbf{Benchmark}}  & \multicolumn{2}{|c|}{\textbf{ZFT-Platzierung}} & \multicolumn{2}{|c|}{\textbf{Zufallsplatzierung}} & \multicolumn{2}{|c|}{\textbf{Differenz (\%)}}\\ \cline{2-7}
                                &   Zeit (ms) & Kosten      &   Zeit (ms)   &    Kosten     &   Zeit    &   Kosten  \\ \hline
                    alu4        &   22347   &   390,582     &   22389       &    391.755    & -0.19     & -0.30   \\ \hline
                    apex2       &   35065   &   538,017     &   31763       &    531.647    & 10.40     & 1.20   \\ \hline    
                    apex4       &   13855   &   310,617     &   13833       &    299.461    & 0.16      & 3.73   \\ \hline        
                    bigkey      &   16534   &   505,461     &   15362       &    515.345    & 7.63      & -1.92   \\ \hline        
                    clma        &   408551  &   3555,85     &   507620      &    3493.84    & -19.52    & 1.77   \\ \hline        
                    des         &   2272    &   368,297     &   1858        &    368.784    & 22.28     & -0.13   \\ \hline        
                    diffeq      &   26617   &   385,295     &   25555       &    385.582    & 4.16      & -0.07   \\ \hline        
                    dsip        &   22176   &   456,205     &   20358       &    460.81     & 8.93      & -1.00   \\ \hline        
                    elliptic    &   346639  &   1256,4      &   319950      &    1314.94    & 8.34      & -4.45   \\ \hline         
                    ex5p        &   18487   &   268,953     &   16410       &    276.82     & 12.66     & -2.84   \\ \hline        
                    ex1010      &   123064  &   1266,36     &   108375      &    1239.9     & 13.55     & 2.13   \\ \hline        
                    frisc       &   138902  &   1245,97     &   142276      &    1359.29    & -2.37     & -8.34   \\ \hline        
                    misex3      &   14292   &   340,409     &   10634       &    337.241    & 34.40     & 0.94   \\ \hline        
                    pdc         &   213463  &   1731,39     &   158013      &    1725.91    & 35.09     & 0.32   \\ \hline        
                    s298        &   50932   &   538,604     &   48419       &    538.008    & 5.19      & 0.11   \\ \hline        
                    s38417      &   210034  &   1977,98     &   187441      &    1893.73    & 12.05     & 4.45   \\ \hline        
                    s38584.1    &   1179389 &   2621,59     &   884384      &    2701.24    & 33.36     & -2.95   \\ \hline        
                    seq         &   33346   &   465,327     &   35338       &    464.718    & -5.64     & 0.13   \\ \hline        
                    spla        &   101775  &   1188,14     &   98363       &    1218.62    & 3.47      & -2.50   \\ \hline        
                    tseng       &   8084    &   213,566     &   8445        &    230.618    & -4.27     & -7.39   \\ \hline        
                \caption{Kostenfunktionen und Laufzeit unterschiedlicher Initialplatzierung}
                \label{tab:init}
            \end{longtable}
        \end{center}



    \section{Platzierungskosten}\label{sec:result-costs}

        In diesem Kapitel werden die Ergebnisse der Platzierungen,
        die durch VPR und durch den entwickelten Algorithmus berechnet wurden, miteinander verglichen.
        Je niedriger die Kosten, desto besser ist das Endergebnis. Die Tabelle (\ref{tab:costs})
        stellt die Ergebnissee der 20 Benchmarks dar. Eine negative Differenz bedeutet,
        dass die VPR Platzierung um den jeweiligen prozentualen Wert weniger Platzierungskosten verursacht.
        Die Ergebnisse der Kräfteplatzierung sind bis zu $\approx 75 \%$
        schlechter als die der VPR-Platzierung.\\
        Dabei ist zu erwähnen, dass VPR einen Algorithmus verwendet
        der auf Simulated annealing zurückzuführen ist.
        Dieser ist ein weit verbreiteter Algorithmus und liefert sehr gute Ergebnisse.
        Dies bedeutet im Umkehrschluss, dass die Ergebnisse schwer miteinander zu vergleichen sind. 

        \begin{center}
            \begin{longtable}{| l | r | r | r |}
                \hline
                \textbf{Benchmark}  & \textbf{ZFT-Kosten} & \textbf{VPR-Kosten} & \textbf{Differenz (\%)} \\
                \hline
                    alu4        &   390,582     &   190,135     &   -51,32 \\ \hline
                    apex2       &   538,017     &   269,765     &   -49,86 \\ \hline    
                    apex4       &   310,617     &   179,329     &   -42,27 \\ \hline        
                    bigkey      &   505,461     &   185,977     &   -63,21 \\ \hline        
                    clma        &   3555,85     &   1387,05     &   -60,99 \\ \hline        
                    des         &   368,297     &   227,843     &   -38,14 \\ \hline        
                    diffeq      &   385,295     &   146,394     &   -62,00 \\ \hline        
                    dsip        &   456,205     &   169,991     &   -62,74 \\ \hline        
                    elliptic    &   1256,4      &   457,203     &   -63,61 \\ \hline         
                    ex5p        &   268,953     &   162,012     &   -39,76 \\ \hline        
                    ex1010      &   1266,36     &   655,429     &   -48,24 \\ \hline        
                    frisc       &   1245,97     &   515,59      &   -58,62 \\ \hline        
                    misex3      &   340,409     &   190,205     &   -44,12 \\ \hline        
                    pdc         &   1731,39     &   898,44      &   -48,11 \\ \hline        
                    s298        &   538,604     &   203,949     &   -62,13 \\ \hline        
                    s38417      &   1977,98     &   671,75      &   -66,04 \\ \hline        
                    s38584.1    &   2621,59     &   657,87      &   -74,91 \\ \hline        
                    seq         &   465,327     &   247,658     &   -46,78 \\ \hline        
                    spla        &   1188,14     &   593,969     &   -50,01 \\ \hline        
                    tseng       &   213,566     &   92,0471     &   -56,90 \\ \hline        
                \caption{Kosten der Platzierungen im Vergleich zu VPR}
                \label{tab:costs}
            \end{longtable}
        \end{center}

    \section{Kritischer Pfad und Kanalbreite}\label{sec:result-routing}

        Im folgenden Kapitel werden die Ergebnisse der Verdrahtungen, die durch VPR durchgeführt wurden,
        miteinander verglichen. Dabei hat VPR jeweils die vorgegebenen Platzierungsbenchmarks sowie die
        berechneten Platzierungen durch die entworfene Software verdrahtet.
        Die wichtigsten Kenngrößen nach dem Verdrahtungsschritt sind die Kanalbreite
        sowie die Dauer des kritischen Pfades. Beide Werte sind je besser,
        desto niedriger sie sind und werden in Tabelle (\ref{tab:routing}) dargestellt.
        Die negativen Werte in der Differenzspalte besagen, dass das VPR-Routing eine
        niedrigere Kanalbreite bzw. einen niedrigeren kritischen Pfad verursacht.
        Dass hierbei das VPR-Routing besser ist, liegt daran, dass wie schon in Kapitel (\ref{sec:result-costs})
        erläutert die Platzierungskosten geringer sind. 
        Dementsprechend kann einfacher und mit niedrigerer Kanalbreite
        geroutet werden. Die Differenz der Kanalbreite hat jedoch keinen direkten Zusammenhang mit der
        Differenz des kritischen Pfades, wie das Beispiel des Benchmarks \textit{bigkey} oder \textit{des} zeigt.
        Obwohl in beiden Fällen die Kanalbreite im ZFT-Routing weit aus höher ist als im VPR-Routing, ist
        die Dauer des kritischen Pfades nur $\approx 8\%$ bzw $\approx 6\%$ niedriger.


        \begin{center}
            \begin{longtable}{| l | r | r | r | r | r | r |}
                \hline
                \multirow{2}*{\textbf{Benchmark}}  & \multicolumn{2}{|c|}{\textbf{ZFT-Routing}} & \multicolumn{2}{|c|}{\textbf{VPR-Routing}} & \multicolumn{2}{|c|}{\textbf{Differenz (\%)}}\\ \cline{2-7}
                                &   Kanalbreite &   K. Pfad     &   Kanalbreite &   K. Pfad  &   Kanalbreite &   K. Pfad\\ \hline
                    alu4        &   22  &   1.54E-07    &   11  &   1.12E-07    &   -50.00  &   -27.50   \\ \hline
                    apex2       &   24  &   2.01E-07    &   12  &   1.41E-07    &   -50.00  &   -29.87   \\ \hline    
                    apex4       &   21  &   1.47E-07    &   13  &   1.31E-07    &   -38.10  &   -10.62   \\ \hline        
                    bigkey      &   23  &   1.14E-07    &   6   &   1.06E-07    &   -73.91  &   -7.40    \\ \hline        
                    clma        &   36  &   4.11E-07    &   13  &   2.54E-07    &   -63.89  &   -38.30   \\ \hline        
                    des         &   12  &   1.29E-07    &   8   &   1.21E-07    &   -33.33  &   -5.66    \\ \hline        
                    diffeq      &   21  &   1.84E-07    &   8   &   8.82E-08    &   -61.90  &   -52.11   \\ \hline        
                    dsip        &   23  &   9.21E-08    &   6   &   7.98E-08    &   -73.91  &   -13.44   \\ \hline        
                    elliptic    &   32  &   2.55E-07    &   11  &   1.83E-07    &   -65.63  &   -28.45   \\ \hline         
                    ex5p        &   22  &   1.29E-07    &   14  &   1.09E-07    &   -36.36  &   -15.16   \\ \hline        
                    ex1010      &   22  &   2.87E-07    &   11  &   2.48E-07    &   -50.00  &   -13.61   \\ \hline        
                    frisc       &   27  &   3.01E-07    &   13  &   1.69E-07    &   -51.85  &   -43.85   \\ \hline        
                    misex3      &   20  &   1.34E-07    &   12  &   1.06E-07    &   -40.00  &   -20.86   \\ \hline        
                    pdc         &   34  &   3.17E-07    &   18  &   2.39E-07    &   -47.06  &   -24.52   \\ \hline        
                    s298        &   22  &   2.97E-07    &   8   &   1.98E-07    &   -63.64  &   -33.33   \\ \hline        
                    s38417      &   26  &   2.81E-07    &   8   &   1.77E-07    &   -69.23  &   -36.83   \\ \hline        
                    s38584.1    &   38  &   2.10E-07    &   9   &   1.27E-07    &   -76.32  &   -39.50   \\ \hline        
                    seq         &   23  &   1.39E-07    &   12  &   1.27E-07    &   -47.83  &   -9.00    \\ \hline        
                    spla        &   28  &   2.64E-07    &   14  &   1.76E-07    &   -50.00  &   -33.29   \\ \hline        
                    tseng       &   17  &   1.09E-07    &   7   &   9.00E-08    &   -58.82  &   -17.57   \\ \hline        
                \caption{Kritischer Pfad und Kanalbreite im Vergleich zu VPR}
                \label{tab:routing}
            \end{longtable}
        \end{center}